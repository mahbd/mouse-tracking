% Options for packages loaded elsewhere
\PassOptionsToPackage{unicode}{hyperref}
\PassOptionsToPackage{hyphens}{url}
%
\documentclass[
  12pt,
]{article}
\usepackage{amsmath,amssymb}
\usepackage{setspace}
\usepackage{iftex}
\ifPDFTeX
  \usepackage[T1]{fontenc}
  \usepackage[utf8]{inputenc}
  \usepackage{textcomp} % provide euro and other symbols
\else % if luatex or xetex
  \usepackage{unicode-math} % this also loads fontspec
  \defaultfontfeatures{Scale=MatchLowercase}
  \defaultfontfeatures[\rmfamily]{Ligatures=TeX,Scale=1}
\fi
\usepackage{lmodern}
\ifPDFTeX\else
  % xetex/luatex font selection
\fi
% Use upquote if available, for straight quotes in verbatim environments
\IfFileExists{upquote.sty}{\usepackage{upquote}}{}
\IfFileExists{microtype.sty}{% use microtype if available
  \usepackage[]{microtype}
  \UseMicrotypeSet[protrusion]{basicmath} % disable protrusion for tt fonts
}{}
\makeatletter
\@ifundefined{KOMAClassName}{% if non-KOMA class
  \IfFileExists{parskip.sty}{%
    \usepackage{parskip}
  }{% else
    \setlength{\parindent}{0pt}
    \setlength{\parskip}{6pt plus 2pt minus 1pt}}
}{% if KOMA class
  \KOMAoptions{parskip=half}}
\makeatother
\usepackage{xcolor}
\usepackage[margin=1in]{geometry}
\usepackage{longtable,booktabs,array}
\usepackage{calc} % for calculating minipage widths
% Correct order of tables after \paragraph or \subparagraph
\usepackage{etoolbox}
\makeatletter
\patchcmd\longtable{\par}{\if@noskipsec\mbox{}\fi\par}{}{}
\makeatother
% Allow footnotes in longtable head/foot
\IfFileExists{footnotehyper.sty}{\usepackage{footnotehyper}}{\usepackage{footnote}}
\makesavenoteenv{longtable}
\setlength{\emergencystretch}{3em} % prevent overfull lines
\providecommand{\tightlist}{%
  \setlength{\itemsep}{0pt}\setlength{\parskip}{0pt}}
\setcounter{secnumdepth}{-\maxdimen} % remove section numbering
\ifLuaTeX
  \usepackage{selnolig}  % disable illegal ligatures
\fi
\usepackage{bookmark}
\IfFileExists{xurl.sty}{\usepackage{xurl}}{} % add URL line breaks if available
\urlstyle{same}
\hypersetup{
  hidelinks,
  pdfcreator={LaTeX via pandoc}}

\author{}
\date{}

\begin{document}

{
\setcounter{tocdepth}{3}
\tableofcontents
}
\setstretch{1.3}
\newpage

hispagestyle\{empty\}

\begin{center}
extbf{UNDERGRADUATE THESIS}\\[1.5cm]

{\LARGE \textbf{Mouse Tracking --- Behavioral Biometrics and Anomaly Detection}}\\[1.2cm]

Submitted by\\[0.6cm]
Mahmudul Alam\\
ID: 1905023 \quad\textbullet\quad Reg. No.: 000012751\\[0.4cm]
Atiqur Rahman\\
ID: 1905005 \quad\textbullet\quad Reg. No.: 000012733\\[1cm]

Department of Computer Science and Engineering\\
[University/Institute Name --- update here]\\
[City, Country]\\[1cm]

Supervised by\\[0.3cm]
[Supervisor Name], [Title]\\
Department of Computer Science and Engineering\\
[University/Institute Name]\\[1.5cm]

August 2025
\end{center}

\newpage

\newpage

\subsection{Declaration}\label{declaration}

We, Mahmudul Alam (ID: 1905023, Reg. No.: 000012751) and Atiqur Rahman
(ID: 1905005, Reg. No.: 000012733), hereby declare that the work
presented in this thesis titled ``Mouse Tracking --- Behavioral
Biometrics and Anomaly Detection'' is the outcome of our own research
carried out under the supervision of the undersigned supervisor(s).

To the best of our knowledge, no part of this thesis has been submitted,
in whole or in part, for any other degree or qualification at this or
any other institution. All sources of information have been duly
acknowledged.

The research adhered to applicable ethical and safety guidelines. Any
risks to participants were minimized and appropriate consent and data
protection measures were applied.

Author: Mahmudul Alam\\
Signature: **********\_\_**********\\
Date: ************\_\_\_\_************

Author: Atiqur Rahman\\
Signature: **********\_\_**********\\
Date: ************\_\_\_\_************

\newpage

\newpage

\subsection{Certificate}\label{certificate}

This is to certify that the thesis entitled ``Mouse Tracking ---
Behavioral Biometrics and Anomaly Detection'' submitted by Mahmudul Alam
(ID: 1905023, Reg. No.: 000012751) and Atiqur Rahman (ID: 1905005, Reg.
No.: 000012733) has been carried out under my supervision. To the best
of my knowledge, this work is original and has not been submitted in
whole or in part for any degree or diploma at this or any other
institution.

The candidates have complied with relevant ethical and safety standards
during the research.

Supervisor: {[}Name{]}, {[}Title{]}\\
Department of Computer Science and Engineering\\
{[}University/Institute Name{]}

Signature: **********\_\_**********\\
Date: ************\_\_\_\_************

\newpage

\newpage

\subsection{Acknowledgements}\label{acknowledgements}

First and foremost, we are grateful to the Almighty for giving us the
strength, health, and perseverance to complete this work.

We sincerely thank our supervisor(s) for their guidance, encouragement,
and constructive feedback throughout this project. Their insights helped
us refine our research questions and methodology, and their constant
support made the journey possible.

We are also thankful to the faculty members and staff of the Department
of Computer Science and Engineering for building the academic foundation
and providing the resources that enabled this work. We appreciate the
help of our peers and friends for fruitful discussions, feedback on
early drafts, and assistance during data collection.

We extend our gratitude to all participants who contributed data for
this study and to the maintainers of the open-source tools and libraries
used in our implementation.

Finally, we owe our deepest thanks to our families for their
unconditional love, patience, and encouragement. Their belief in us has
been our constant motivation.

\newpage

\section{Abstract}\label{abstract}

We study mouse tracking as a behavioral biometric for user
identification and anomaly detection. Using 76,693 behavioral segments
from four users, we engineer motion-, speed-, and acceleration-based
features from fixed 50-event windows. Random Forest reaches 85.36\%
classification accuracy. For single-user anomaly detection, One-Class
SVM and Isolation Forest calibrated at 5\% contamination achieve
expected self-test anomaly rates and reveal substantial cross-user
differences---up to 31.6\% anomalies---demonstrating distinctive
behavioral signatures. We detail a practical end-to-end system with
cross-platform collectors, a feature pipeline, model training scripts,
and an optional Windows GUI for real-time detection. We discuss
limitations and ethical aspects and outline directions for scaling,
robustness, and privacy-preserving deployment.

\section{Introduction}\label{introduction}

Behavioral biometrics leverage the unique ways individuals interact with
computing systems. Mouse tracking---capturing sequences of cursor
movements, clicks, and scroll events---offers a continuous, unobtrusive
signal for identifying users and detecting anomalous behavior. Prior
studies have explored mouse-based authentication and region usage
patterns in human-computer interaction contexts \cite{rahman2021}. This
thesis investigates mouse tracking as a behavioral biometric for two
goals: multi-user classification (who is at the computer) and
single-user anomaly detection (is the current behavior unusual for this
user?).

\subsection{Motivation}\label{motivation}

\begin{itemize}
\tightlist
\item
  Security: Continuous authentication complements point-in-time logins.
\item
  Usability: Passive and privacy-conscious when engineered from motion
  dynamics rather than content.
\item
  Practicality: Mouse signals are ubiquitous across desktop environments
  and inexpensive to collect.
\end{itemize}

\subsection{Objectives}\label{objectives}

\begin{enumerate}
\def\labelenumi{\arabic{enumi}.}
\tightlist
\item
  Design and implement a cross-platform data collection and feature
  extraction pipeline for mouse events.
\item
  Engineer robust behavioral features from fixed-length event segments.
\item
  Train and compare classification models for user identification.
\item
  Train One-Class SVM and Isolation Forest for single-user anomaly
  detection.
\item
  Evaluate cross-user anomaly rates to quantify behavioral
  distinctiveness.
\end{enumerate}

\subsection{Contributions}\label{contributions}

\begin{itemize}
\tightlist
\item
  An end-to-end open codebase with C++ collectors (Windows, Linux),
  Python preprocessing, classic ML baselines, and a minimal GUI for
  real-time anomaly detection.
\item
  A curated feature set centered on motion, speed/acceleration
  statistics, and event pattern ratios.
\item
  A comprehensive experimental study on 76,693 behavioral segments
  across four users, achieving 85.36\% best classification accuracy
  (Random Forest) and notable cross-user anomaly separation.
\end{itemize}

\subsection{Thesis Outline}\label{thesis-outline}

\begin{itemize}
\tightlist
\item
  Background and Related Work: Behavioral biometrics, mouse dynamics,
  anomaly detection.
\item
  Data and Feature Engineering: Raw signal, segmentation, features,
  scaling, exclusions.
\item
  Methodology: Modeling tasks, algorithms, training protocols,
  validation.
\item
  System Implementation: Collectors, preprocessing, models, GUI app,
  storage layout.
\item
  Experiments and Results: Dataset, metrics, classifier performance,
  anomaly detection analyses.
\item
  Discussion and Future Work: Limitations, threats to validity, ethical
  considerations, next steps.
\item
  Conclusion: Findings, implications, and closing remarks.
\end{itemize}

Note: Results summarized here are derived from
\texttt{results/results.md} and \texttt{results/training\_results.txt}
within this repository; reproducibility details are provided in the
appendices.

\subsection{Motivation and Scope}\label{motivation-and-scope}

Security systems increasingly rely on continuous authentication to
supplement one-time logins. Mouse dynamics provide an unobtrusive,
evergreen signal that can be collected without specialized hardware.
This work focuses on everyday, free-form computer use rather than
constrained gestures. The scope includes feature engineering from raw
events, classical ML baselines for user identification, and single-user
anomaly detection for continuous authentication.

We center our analysis on a four-user dataset (76,693 segments) with
fixed 50-event windows, representative of a modest but realistic
behavioral corpus. The methodology is designed to scale to more users
and to integrate with additional modalities.

\subsection{Contributions and Outline}\label{contributions-and-outline}

\subsubsection{Contributions}\label{contributions-1}

\begin{itemize}
\tightlist
\item
  End-to-end open implementation: multi-OS collectors, feature pipeline,
  models, and a lightweight GUI.
\item
  Practical feature set capturing speed, acceleration, trajectory
  geometry, and event ratios.
\item
  Empirical study showing 85.36\% classification accuracy and meaningful
  cross-user anomaly separation.
\end{itemize}

\subsubsection{Outline}\label{outline}

\begin{enumerate}
\def\labelenumi{\arabic{enumi}.}
\tightlist
\item
  Background and Related Work
\item
  Data and Feature Engineering
\item
  Methodology
\item
  System Implementation
\item
  Experiments and Results
\item
  Discussion and Future Work
\item
  Conclusion
\end{enumerate}

\section{Background and Related Work}\label{background-and-related-work}

This section surveys behavioral biometrics with emphasis on mouse
dynamics, anomaly detection, and user classification.

\subsection{Behavioral Biometrics}\label{behavioral-biometrics}

Behavioral biometrics analyze how people perform actions (typing, moving
a mouse, swiping, gait). Unlike physiological traits (fingerprint,
face), behavioral traits can be monitored continuously and updated over
time, enabling continuous authentication and intrusion detection.

\subsection{Mouse Dynamics}\label{mouse-dynamics}

Prior work shows that cursor trajectories, speed/acceleration profiles,
and click timing can uniquely characterize users \cite{rahman2021}.
Common paradigms include:

\begin{itemize}
\tightlist
\item
  Fixed-task gestures (e.g., target acquisition, drag-and-drop)
\item
  Free-form activity logs during normal computer use
\end{itemize}

Feature classes often include trajectory geometry, kinematics, frequency
of event types, and context (application window, time-of-day). Our
pipeline focuses on free-form events aggregated into fixed-length
segments.

\subsection{Anomaly Detection
Techniques}\label{anomaly-detection-techniques}

\begin{itemize}
\tightlist
\item
  One-Class SVM: Learns a decision boundary around normal data using an
  RBF kernel. Key parameter: nu (anomaly fraction). Pros: solid
  theoretical foundation; Cons: scaling-sensitive.
\item
  Isolation Forest: Isolates anomalies via random partitions. Key
  parameter: contamination. Pros: efficient, less sensitive to scaling;
  Cons: may over-flag in heterogeneous data.
\end{itemize}

\subsection{User Classification}\label{user-classification}

Classical ML (Random Forest, Decision Trees, KNN, Naive Bayes) remains
competitive on tabular behavioral features. Ensembles capture non-linear
interactions and often outperform linear baselines. Neural networks can
underperform without abundant data and tuning.

\subsection{Threats and
Considerations}\label{threats-and-considerations}

\begin{itemize}
\tightlist
\item
  Data drift over time (fatigue, device change) threatens stability.
\item
  Privacy: mouse dynamics are less sensitive than content, but still
  identifying. Aggregation and differential privacy can mitigate risks.
\item
  Adversarial concerns: spoofing trajectories may be possible;
  continuous monitoring and multi-modal fusion improve robustness.
\end{itemize}

\subsection{References}\label{references}

\begin{itemize}
\tightlist
\item
  \cite{rahman2021}: Mouse movement-driven authentication and region
  usage (IEEE CCWC 2021)
\item
  \cite{paper_document}: Additional PDF resource provided with limited
  metadata; update when full details are known
\end{itemize}

\subsection{Behavioral Biometrics
Survey}\label{behavioral-biometrics-survey}

Behavioral biometrics encompasses keystroke dynamics, mouse and touch
gestures, gaze, gait, and stylometry. Strengths include continual
monitoring and liveness; challenges involve variability due to fatigue,
stress, or device changes. Mouse dynamics sit at the intersection of
usability and security, providing a rich signal from routine
interactions.

\subsection{Mouse Dynamics Literature}\label{mouse-dynamics-literature}

Early work leveraged constrained tasks; more recent studies mine
free-form usage data. Features typically include kinematics (speed,
acceleration), curvature/jerk, pause timing, and click sequences.
Classifiers range from SVM and Random Forests to HMMs and deep models,
though deep models require large datasets. Our approach favors
feature-rich, classic ML due to data scale.

See \cite{rahman2021} for an example of authentication and
region-of-interest inference from mouse movement data.

\section{Data and Feature
Engineering}\label{data-and-feature-engineering}

\subsection{Raw Mouse Event Data}\label{raw-mouse-event-data}

Raw CSV events include: WindowTitle (hash or 0), State
(DM/VM/HM/LD/LU/RD/RU/MW), Time Diff (ms), Day Time (5-min bin), X Pos,
Y Pos.

\subsection{Segmentation Strategy}\label{segmentation-strategy}

\begin{itemize}
\tightlist
\item
  Fixed-length windows of 50 consecutive events per behavioral segment.
\item
  Benefits: controls for window size variability; simplifies model
  input.
\item
  Alternatives: time-based windows, overlapping windows (future work).
\end{itemize}

\subsection{Engineered Features}\label{engineered-features}

We compute 36 features and typically select a 16-feature subset for
modeling:

\begin{itemize}
\tightlist
\item
  Temporal: segment\_duration\_ms
\item
  Spatial: total\_distance\_pixels, path\_straightness
\item
  Speed: mean\_speed, std\_dev\_speed, median\_speed, skewness\_speed,
  kurtosis\_speed, max\_speed, min\_speed
\item
  Acceleration: mean\_acceleration, std\_dev\_acceleration,
  max\_acceleration
\item
  Movement ratios: ratio\_DM, ratio\_VM, ratio\_HM
\item
  Context/identity: user, most\_common\_window\_title\_hash,
  most\_common\_daytime\_bin, std\_dev\_daytime\_bin, event counts
  (often excluded)
\end{itemize}

\subsection{Preprocessing}\label{preprocessing}

\begin{itemize}
\tightlist
\item
  Numeric coercion for timing fields; drop invalid rows.
\item
  Feature exclusions: remove counts and metadata for behavior-only
  models.
\item
  Standardization: StandardScaler fit per training split; scalers saved
  to \texttt{models/scalers/}.
\end{itemize}

\subsection{Datasets}\label{datasets}

\begin{itemize}
\tightlist
\item
  Users: atiq, masum, rakib, zia
\item
  Total segments: 76,693
\item
  Distribution per user given in \texttt{results/results.md}.
\end{itemize}

\subsection{Storage Layout}\label{storage-layout}

\begin{itemize}
\tightlist
\item
  Processed per-user features:
  \texttt{processed/\textless{}username\textgreater{}.csv}
\item
  Unified features for classification: \texttt{processed/features.csv}
\end{itemize}

\subsection{Reproducibility Notes}\label{reproducibility-notes}

\begin{itemize}
\tightlist
\item
  Preprocess with \texttt{collection/preprocess.py}.
\item
  Keep segmentation length and feature list consistent between training
  and inference (GUI).
\end{itemize}

\subsection{Segmentation and Windows}\label{segmentation-and-windows}

We segment raw event streams into fixed 50-event windows to standardize
sample length and simplify batch processing. Alternatives include
time-based windows and overlapping strides. Fixed windows reduce
complexity and encourage consistent feature distributions, at the cost
of potential boundary effects.

\subsection{Feature Definitions}\label{feature-definitions}

\begin{itemize}
\tightlist
\item
  segment\_duration\_ms: sum of time deltas within the window.
\item
  total\_distance\_pixels: cumulative Euclidean distance along cursor
  positions.
\item
  path\_straightness: ratio of end-to-end distance to path length
  (0,1{]}.
\item
  mean\_speed, std\_dev\_speed, median\_speed: first-order kinematics
  from distance/time.
\item
  skewness\_speed, kurtosis\_speed: higher-moment descriptors of speed
  distribution.
\item
  max\_speed, min\_speed: extremes of estimated speed.
\item
  mean\_acceleration, std\_dev\_acceleration, max\_acceleration: changes
  in speed per unit time.
\item
  ratio\_DM/VM/HM: fraction of movement events with
  diagonal/vertical/horizontal components.
\end{itemize}

Counts and identity-related features (e.g., window title hash) are often
excluded in modeling to focus on behavior.

\section{Methodology}\label{methodology}

\subsection{Problem Formulations}\label{problem-formulations}

\begin{itemize}
\tightlist
\item
  Multi-user classification: Predict \texttt{user} label from behavioral
  segment features.
\item
  Single-user anomaly detection: Learn ``normal'' behavior for a user
  and flag deviations.
\end{itemize}

\subsection{Algorithms}\label{algorithms}

\begin{itemize}
\tightlist
\item
  Classification: Random Forest, Decision Tree, KNN, Naive Bayes,
  PCA+XGBoost, MLP.
\item
  Anomaly Detection: One-Class SVM (RBF, nu=0.05), Isolation Forest
  (n\_estimators=100, contamination=0.05, random\_state=42).
\end{itemize}

\subsection{Training Protocols}\label{training-protocols}

\begin{itemize}
\tightlist
\item
  Classification: 5-fold stratified cross-validation; report mean
  accuracy and per-user precision/recall/F1 on held-out folds.
\item
  Anomaly: Train on each user's feature file; validate expected anomaly
  fraction on training set; perform cross-user tests using other users'
  data.
\end{itemize}

\subsection{Feature Selection}\label{feature-selection}

\begin{itemize}
\tightlist
\item
  Use 16 core behavioral features; exclude counts and direct identity
  cues.
\item
  Scale numerical features with StandardScaler; persist scaler alongside
  model.
\end{itemize}

\subsection{Evaluation Metrics}\label{evaluation-metrics}

\begin{itemize}
\tightlist
\item
  Classification: Accuracy, precision, recall, F1-score.
\item
  Anomaly: Anomaly rate (\% flagged) on self and cross-user data.
\end{itemize}

\subsection{Validation and
Significance}\label{validation-and-significance}

\begin{itemize}
\tightlist
\item
  Consistent preprocessing across models.
\item
  Repeatability ensured by fixed random seeds where applicable.
\end{itemize}

\subsection{Implementation Details}\label{implementation-details}

\begin{itemize}
\tightlist
\item
  Hyperparameter search: GridSearchCV for RF (trees, depth, features).
  Fixed params for SVM/ISO for comparability.
\item
  Data splitting: Stratified folds preserve per-user class balance.
\item
  Scaling: Fit scaler on training folds only; apply to validation folds
  to avoid leakage.
\end{itemize}

\subsection{Threats to Validity and
Mitigation}\label{threats-to-validity-and-mitigation}

\begin{itemize}
\tightlist
\item
  Small sample of users: address through cross-user tests and
  transparent reporting.
\item
  Potential device variance: note collector environment; future work
  includes device-normalized features.
\item
  Temporal drift: limited in current dataset; propose longitudinal
  validation.
\end{itemize}

\subsection{Algorithms and
Hyperparameters}\label{algorithms-and-hyperparameters}

\begin{itemize}
\tightlist
\item
  Random Forest: n\_estimators, max\_depth, max\_features tuned by CV.
\item
  Decision Tree: depth/pruning parameters to manage overfitting.
\item
  KNN: k and distance metric; sensitive to scaling.
\item
  Naive Bayes: Gaussian assumption on standardized features.
\item
  PCA+XGBoost: reduce to \textasciitilde50\% variance dimensions;
  gradient-boosted trees thereafter.
\item
  MLP: modest hidden layers; requires larger datasets for optimal
  results.
\item
  One-Class SVM: RBF kernel, nu=0.05, gamma=scale.
\item
  Isolation Forest: n\_estimators=100, contamination=0.05,
  random\_state=42.
\end{itemize}

\subsection{Evaluation and Metrics}\label{evaluation-and-metrics}

\begin{itemize}
\tightlist
\item
  Classification: accuracy, precision, recall, F1; per-class support;
  confusion matrix.
\item
  Anomaly detection: fraction flagged anomalous; cross-user anomaly
  rates.
\item
  Calibration: verify self-test anomaly rate matches contamination/nu;
  adjust thresholds as needed for deployment.
\item
  Robustness checks: sensitivity to segmentation length and feature
  subsets.
\end{itemize}

\section{System Implementation}\label{system-implementation}

\subsection{Overview}\label{overview}

The system integrates data collection, preprocessing/feature
engineering, model training, and an optional real-time GUI app.

\subsection{Collectors (C++)}\label{collectors-c}

\begin{itemize}
\tightlist
\item
  Windows: \texttt{collection/collector.cpp} and
  \texttt{AnomalyDetectorApp/mouse\_logger.cpp} (low-level hook).
\item
  Linux/Wayland: \texttt{collection/collector\_linux.cpp} using
  libinput/udev; root privileges required.
\end{itemize}

\subsection{Preprocessing (Python)}\label{preprocessing-python}

\begin{itemize}
\tightlist
\item
  Script: \texttt{collection/preprocess.py} segments events (50 per
  window) and computes features.
\item
  Outputs: \texttt{processed/\textless{}username\textgreater{}.csv} per
  user; merged \texttt{processed/features.csv} for classification.
\end{itemize}

\subsection{Modeling (Python)}\label{modeling-python}

\begin{itemize}
\tightlist
\item
  Classification: \texttt{classification/*.py} scripts implement
  baselines and an MLP.
\item
  Anomaly: \texttt{abnormal/one\_class\_svm.py},
  \texttt{abnormal/isolation\_forest.py} with predictors.
\item
  Models and scalers saved to \texttt{models/} subfolders.
\end{itemize}

\subsection{Real-Time GUI}\label{real-time-gui}

\begin{itemize}
\tightlist
\item
  \texttt{AnomalyDetectorApp/main\_app.py} reads events from the Windows
  logger, batches to segments, scales, and applies a trained One-Class
  SVM.
\item
  Configuration: BATCH\_SIZE, SEGMENT\_LENGTH\_EVENTS, and
  TRAINING\_FEATURES must match training setup.
\end{itemize}

\subsection{Project Structure}\label{project-structure}

See repository \texttt{README.md} for a detailed overview of folders and
scripts.

\subsection{Collectors}\label{collectors}

\begin{itemize}
\tightlist
\item
  Windows low-level hook logger outputs event streams for the GUI.
\item
  Linux/Wayland collector uses libinput/udev to write periodic CSVs.
\item
  Design goals: minimal overhead, stable timestamps, and portability.
\end{itemize}

\subsection{Preprocessing and
Features}\label{preprocessing-and-features}

The Python pipeline reads raw CSVs, segments windows, and computes
features. It ensures numeric coercion, handles missing values via row
drops, and persists per-user feature files. Consistency between training
and inference is enforced via shared TRAINING\_FEATURES definitions and
saved scalers.

\section{Experiments and Results}\label{experiments-and-results}

This section consolidates findings from \texttt{results/results.md} and
\texttt{results/training\_results.txt}.

\subsection{Dataset Summary}\label{dataset-summary}

\begin{itemize}
\tightlist
\item
  76,693 segments across 4 users: atiq, masum, rakib, zia.
\item
  36 engineered features; 16 core features used for modeling.
\item
  Segmentation: 50 events per segment.
\end{itemize}

\subsection{Classification
Performance}\label{classification-performance}

\begin{itemize}
\tightlist
\item
  Random Forest (best): 85.36\% accuracy overall.
\item
  Decision Tree: 77.24\%
\item
  PCA + XGBoost: 70.20\%
\item
  KNN: 60.30\%
\item
  MLP: 44.43\%
\item
  Naive Bayes: 38.37\%
\end{itemize}

Per-user (Random Forest): masum highest precision/recall (≈98\%);
rakib/zia show more confusion.

\subsection{Anomaly Detection}\label{anomaly-detection}

\begin{itemize}
\tightlist
\item
  Self-tests: \textasciitilde5\% anomalies for both One-Class SVM and
  Isolation Forest on their training users (nu/contamination=0.05).
\end{itemize}

Cross-user anomaly rates (selected):

\begin{itemize}
\tightlist
\item
  Masum's models: up to 31.6\% anomalies on other users (most
  distinctive).
\item
  Atiq's models: \textasciitilde1--5\% anomalies on others (least
  distinctive).
\item
  Rakib/Zia: intermediate distinctiveness.
\end{itemize}

Isolation Forest generally yields higher cross-user anomaly rates than
One-Class SVM, indicating greater sensitivity.

\subsection{Key Findings}\label{key-findings}

\begin{itemize}
\tightlist
\item
  Mouse dynamics enable feasible user identification (85.36\% accuracy).
\item
  Behavioral distinctiveness varies by user; useful for continuous
  authentication.
\item
  Ensemble methods outperform simple baselines; deep models need more
  data/tuning.
\end{itemize}

\subsection{Reproducibility}\label{reproducibility}

\begin{itemize}
\tightlist
\item
  See \texttt{appendices/B-reproducibility.md} for environment and
  script details.
\end{itemize}

\subsection{Additional Analyses
(Placeholders)}\label{additional-analyses-placeholders}

\subsubsection{Feature Distributions}\label{feature-distributions}

Figure: Histogram/violin plots for key features (e.g., mean\_speed,
path\_straightness) per user to visualize separability.

\subsubsection{Confusion Matrix (Random
Forest)}\label{confusion-matrix-random-forest}

Figure: 4x4 confusion matrix highlighting misclassifications between
rakib and zia.

\subsubsection{Cross-User Anomaly ROC-style
View}\label{cross-user-anomaly-roc-style-view}

Figure: For each user's model, plot anomaly rate vs.~threshold to
illustrate sensitivity (proxy ROC since labels are cross-user).

\subsubsection{Ablation Study (Planned)}\label{ablation-study-planned}

Table: Impact of removing feature groups (speed stats, ratios,
acceleration) on RF accuracy and ISO cross-user anomaly rates.

\subsection{Classification Details}\label{classification-details}

We trained classic classifiers on \texttt{processed/features.csv} using
stratified 5-fold CV. Random Forest provided the best accuracy
(85.36\%), consistent with non-linear interactions among kinematic and
pattern features. KNN showed sensitivity to scaling and feature
correlation. The MLP underperformed likely due to limited data and
minimal tuning.

\subsection{Anomaly Detection Details}\label{anomaly-detection-details}

For each user, we trained One-Class SVM and Isolation Forest on that
user's segments with 5\% contamination (nu=0.05). Self-tests confirmed
calibration (\textasciitilde5\% anomalies). Cross-user tests quantified
distinctiveness, with Masum's models flagging 11--31.6\% anomalies for
other users, while Atiq's models flagged \textasciitilde1--5\%.

These results indicate varying uniqueness in behavioral patterns and
justify personalized thresholds in deployment.

\section{Tables and Figures}\label{tables-and-figures}

\subsection{Classification Summary
Table}\label{classification-summary-table}

\begin{longtable}[]{@{}ll@{}}
\toprule\noalign{}
Algorithm & Accuracy \\
\midrule\noalign{}
\endhead
\bottomrule\noalign{}
\endlastfoot
Random Forest & 85.36\% \\
Decision Tree & 77.24\% \\
PCA + XGBoost & 70.20\% \\
KNN & 60.30\% \\
MLP & 44.43\% \\
Naive Bayes & 38.37\% \\
\end{longtable}

\subsection{Random Forest Per-User
Metrics}\label{random-forest-per-user-metrics}

\begin{longtable}[]{@{}llll@{}}
\toprule\noalign{}
User & Precision & Recall & F1-Score \\
\midrule\noalign{}
\endhead
\bottomrule\noalign{}
\endlastfoot
atiq & 0.89 & 0.86 & 0.87 \\
masum & 0.98 & 0.98 & 0.98 \\
rakib & 0.79 & 0.82 & 0.81 \\
zia & 0.81 & 0.80 & 0.80 \\
\end{longtable}

\subsection{Anomaly Detection Cross-User
Summary}\label{anomaly-detection-cross-user-summary}

Isolation Forest generally shows higher anomaly rates than One-Class
SVM.

\begin{itemize}
\tightlist
\item
  Masum's models: 11--31.6\% anomalies on others.
\item
  Rakib's models: 2--21.2\%.
\item
  Zia's models: 2--19\%.
\item
  Atiq's models: 1--5.3\%.
\end{itemize}

\begin{quote}
Source: results/training\_results.txt
\end{quote}

\section{Discussion and Future Work}\label{discussion-and-future-work}

\subsection{Interpretation}\label{interpretation}

\begin{itemize}
\tightlist
\item
  High RF accuracy suggests non-linear feature interactions capture
  user-specific kinematics.
\item
  Cross-user anomaly patterns quantify distinctiveness and inform
  threshold selection.
\end{itemize}

\subsection{Limitations}\label{limitations}

\begin{itemize}
\tightlist
\item
  4 users: limited external validity; potential sampling bias.
\item
  Short horizon: temporal stability not assessed.
\item
  Device and environment variability not controlled across diverse
  hardware.
\end{itemize}

\subsection{Ethical and Privacy
Considerations}\label{ethical-and-privacy-considerations}

\begin{itemize}
\tightlist
\item
  Avoid logging content; focus on dynamics and coarse context.
\item
  Consider consent, transparency, and data minimization.
\item
  Explore privacy-preserving training (federated learning, DP
  mechanisms).
\end{itemize}

\subsection{Future Work}\label{future-work}

\begin{itemize}
\tightlist
\item
  Scale to larger, longitudinal datasets.
\item
  Robustness: domain adaptation across devices and tasks.
\item
  Multi-modal fusion with keystroke dynamics and application context.
\item
  Real-time deployment studies; calibration for false positive control.
\item
  Explainability: feature attribution to validate behavioral hypotheses.
\end{itemize}

\subsection{Limitations}\label{limitations-1}

\begin{itemize}
\tightlist
\item
  Small number of users limits generalization.
\item
  Short collection period precludes longitudinal stability analysis.
\item
  Hardware variance and environment factors are not exhaustively
  controlled.
\end{itemize}

\subsection{Ethics and Privacy}\label{ethics-and-privacy}

We minimize privacy risk by focusing on motion dynamics rather than
content. Participants should be informed about collection and usage.
Access controls and retention limits reduce risk. Future directions
include on-device processing and privacy-preserving learning techniques.

\section{Conclusion}\label{conclusion}

This thesis demonstrates that mouse tracking behavioral features support
both user identification and anomaly detection. On a dataset of 76,693
segments from four users, Random Forest achieved 85.36\% accuracy for
classification. One-Class SVM and Isolation Forest produced expected
\textasciitilde5\% anomaly rates on self-tests and revealed meaningful
cross-user differences, with Isolation Forest generally more sensitive.

These results support the viability of continuous, unobtrusive
behavioral authentication. With larger and more diverse datasets,
multi-modal fusion, and longitudinal evaluation, mouse dynamics can
contribute to practical, privacy-conscious security systems.

\section{Appendix A: Dataset Details}\label{appendix-a-dataset-details}

\begin{itemize}
\tightlist
\item
  Users: atiq, masum, rakib, zia.
\item
  Segmentation: 50 events per segment.
\item
  Total segments: 76,693.
\item
  Per-user distribution provided in \texttt{results/results.md}.
\item
  Event definitions: DM, VM, HM, LD/LU, RD/RU, MW.
\item
  Collection environments: Windows and Linux/Wayland collectors.
\end{itemize}

Notes:

\begin{itemize}
\tightlist
\item
  Day Time is encoded as 5-minute buckets (0--287).
\item
  On Linux/Wayland, WindowTitle is 0 due to lack of API access.
\end{itemize}

\section{Appendix B: Reproducibility}\label{appendix-b-reproducibility}

\subsection{Environment}\label{environment}

\begin{itemize}
\tightlist
\item
  Python 3.13
\item
  Key libraries: scikit-learn, pandas, numpy, joblib, xgboost,
  (optional) PyTorch.
\item
  Install with \texttt{pip\ install\ -r\ requirements.txt} in a virtual
  environment.
\end{itemize}

\subsection{Steps}\label{steps}

\begin{enumerate}
\def\labelenumi{\arabic{enumi}.}
\tightlist
\item
  Collect raw CSVs with the OS-specific collectors.
\item
  Run \texttt{collection/preprocess.py} to generate
  \texttt{processed/\textless{}username\textgreater{}.csv}.
\item
  For classification: prepare \texttt{processed/features.csv} with
  multiple users.
\item
  Run scripts in \texttt{classification/} to train and evaluate
  classifiers.
\item
  For anomaly detection: run \texttt{abnormal/one\_class\_svm.py} and
  \texttt{abnormal/isolation\_forest.py} per user, then
  \texttt{predict\_*} scripts for evaluation.
\item
  Real-time: run \texttt{AnomalyDetectorApp/main\_app.py} on Windows
  with a trained SVM model and scaler.
\end{enumerate}

\subsection{Models and Artifacts}\label{models-and-artifacts}

\begin{itemize}
\tightlist
\item
  Models saved in \texttt{models/svm/} and
  \texttt{models/isolation\_forest/}.
\item
  Scalers saved in \texttt{models/scalers/}.
\item
  Results summarized in \texttt{results/}.
\end{itemize}

\subsection{Notes}\label{notes}

\begin{itemize}
\tightlist
\item
  Keep feature lists and scaler alignment consistent between training
  and inference.
\item
  Random seeds are set where applicable for repeatability.
\end{itemize}

\section{Appendix C: Ethics and
Privacy}\label{appendix-c-ethics-and-privacy}

\begin{itemize}
\tightlist
\item
  Data minimization: store only motion dynamics and coarse context.
\item
  Consent and transparency: inform participants of collection and usage.
\item
  Access control and retention: restrict who can access raw logs; set
  deletion policies.
\item
  Safeguards: consider anonymization of window titles via hashing;
  prefer on-device processing.
\item
  Compliance: align with institutional review processes and applicable
  regulations.
\end{itemize}

\end{document}
